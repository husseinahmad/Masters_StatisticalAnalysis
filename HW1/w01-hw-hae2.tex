\documentclass[]{article}
\usepackage{lmodern}
\usepackage{amssymb,amsmath}
\usepackage{ifxetex,ifluatex}
\usepackage{fixltx2e} % provides \textsubscript
\ifnum 0\ifxetex 1\fi\ifluatex 1\fi=0 % if pdftex
  \usepackage[T1]{fontenc}
  \usepackage[utf8]{inputenc}
\else % if luatex or xelatex
  \ifxetex
    \usepackage{mathspec}
  \else
    \usepackage{fontspec}
  \fi
  \defaultfontfeatures{Ligatures=TeX,Scale=MatchLowercase}
\fi
% use upquote if available, for straight quotes in verbatim environments
\IfFileExists{upquote.sty}{\usepackage{upquote}}{}
% use microtype if available
\IfFileExists{microtype.sty}{%
\usepackage{microtype}
\UseMicrotypeSet[protrusion]{basicmath} % disable protrusion for tt fonts
}{}
\usepackage[margin=1in]{geometry}
\usepackage{hyperref}
\PassOptionsToPackage{usenames,dvipsnames}{color} % color is loaded by hyperref
\hypersetup{unicode=true,
            pdftitle={Week 1 - Homework},
            pdfauthor={STAT 420, Summer 2018, Unger},
            colorlinks=true,
            linkcolor=Maroon,
            citecolor=Blue,
            urlcolor=cyan,
            breaklinks=true}
\urlstyle{same}  % don't use monospace font for urls
\usepackage{color}
\usepackage{fancyvrb}
\newcommand{\VerbBar}{|}
\newcommand{\VERB}{\Verb[commandchars=\\\{\}]}
\DefineVerbatimEnvironment{Highlighting}{Verbatim}{commandchars=\\\{\}}
% Add ',fontsize=\small' for more characters per line
\usepackage{framed}
\definecolor{shadecolor}{RGB}{248,248,248}
\newenvironment{Shaded}{\begin{snugshade}}{\end{snugshade}}
\newcommand{\KeywordTok}[1]{\textcolor[rgb]{0.13,0.29,0.53}{\textbf{#1}}}
\newcommand{\DataTypeTok}[1]{\textcolor[rgb]{0.13,0.29,0.53}{#1}}
\newcommand{\DecValTok}[1]{\textcolor[rgb]{0.00,0.00,0.81}{#1}}
\newcommand{\BaseNTok}[1]{\textcolor[rgb]{0.00,0.00,0.81}{#1}}
\newcommand{\FloatTok}[1]{\textcolor[rgb]{0.00,0.00,0.81}{#1}}
\newcommand{\ConstantTok}[1]{\textcolor[rgb]{0.00,0.00,0.00}{#1}}
\newcommand{\CharTok}[1]{\textcolor[rgb]{0.31,0.60,0.02}{#1}}
\newcommand{\SpecialCharTok}[1]{\textcolor[rgb]{0.00,0.00,0.00}{#1}}
\newcommand{\StringTok}[1]{\textcolor[rgb]{0.31,0.60,0.02}{#1}}
\newcommand{\VerbatimStringTok}[1]{\textcolor[rgb]{0.31,0.60,0.02}{#1}}
\newcommand{\SpecialStringTok}[1]{\textcolor[rgb]{0.31,0.60,0.02}{#1}}
\newcommand{\ImportTok}[1]{#1}
\newcommand{\CommentTok}[1]{\textcolor[rgb]{0.56,0.35,0.01}{\textit{#1}}}
\newcommand{\DocumentationTok}[1]{\textcolor[rgb]{0.56,0.35,0.01}{\textbf{\textit{#1}}}}
\newcommand{\AnnotationTok}[1]{\textcolor[rgb]{0.56,0.35,0.01}{\textbf{\textit{#1}}}}
\newcommand{\CommentVarTok}[1]{\textcolor[rgb]{0.56,0.35,0.01}{\textbf{\textit{#1}}}}
\newcommand{\OtherTok}[1]{\textcolor[rgb]{0.56,0.35,0.01}{#1}}
\newcommand{\FunctionTok}[1]{\textcolor[rgb]{0.00,0.00,0.00}{#1}}
\newcommand{\VariableTok}[1]{\textcolor[rgb]{0.00,0.00,0.00}{#1}}
\newcommand{\ControlFlowTok}[1]{\textcolor[rgb]{0.13,0.29,0.53}{\textbf{#1}}}
\newcommand{\OperatorTok}[1]{\textcolor[rgb]{0.81,0.36,0.00}{\textbf{#1}}}
\newcommand{\BuiltInTok}[1]{#1}
\newcommand{\ExtensionTok}[1]{#1}
\newcommand{\PreprocessorTok}[1]{\textcolor[rgb]{0.56,0.35,0.01}{\textit{#1}}}
\newcommand{\AttributeTok}[1]{\textcolor[rgb]{0.77,0.63,0.00}{#1}}
\newcommand{\RegionMarkerTok}[1]{#1}
\newcommand{\InformationTok}[1]{\textcolor[rgb]{0.56,0.35,0.01}{\textbf{\textit{#1}}}}
\newcommand{\WarningTok}[1]{\textcolor[rgb]{0.56,0.35,0.01}{\textbf{\textit{#1}}}}
\newcommand{\AlertTok}[1]{\textcolor[rgb]{0.94,0.16,0.16}{#1}}
\newcommand{\ErrorTok}[1]{\textcolor[rgb]{0.64,0.00,0.00}{\textbf{#1}}}
\newcommand{\NormalTok}[1]{#1}
\usepackage{graphicx,grffile}
\makeatletter
\def\maxwidth{\ifdim\Gin@nat@width>\linewidth\linewidth\else\Gin@nat@width\fi}
\def\maxheight{\ifdim\Gin@nat@height>\textheight\textheight\else\Gin@nat@height\fi}
\makeatother
% Scale images if necessary, so that they will not overflow the page
% margins by default, and it is still possible to overwrite the defaults
% using explicit options in \includegraphics[width, height, ...]{}
\setkeys{Gin}{width=\maxwidth,height=\maxheight,keepaspectratio}
\IfFileExists{parskip.sty}{%
\usepackage{parskip}
}{% else
\setlength{\parindent}{0pt}
\setlength{\parskip}{6pt plus 2pt minus 1pt}
}
\setlength{\emergencystretch}{3em}  % prevent overfull lines
\providecommand{\tightlist}{%
  \setlength{\itemsep}{0pt}\setlength{\parskip}{0pt}}
\setcounter{secnumdepth}{0}
% Redefines (sub)paragraphs to behave more like sections
\ifx\paragraph\undefined\else
\let\oldparagraph\paragraph
\renewcommand{\paragraph}[1]{\oldparagraph{#1}\mbox{}}
\fi
\ifx\subparagraph\undefined\else
\let\oldsubparagraph\subparagraph
\renewcommand{\subparagraph}[1]{\oldsubparagraph{#1}\mbox{}}
\fi

%%% Use protect on footnotes to avoid problems with footnotes in titles
\let\rmarkdownfootnote\footnote%
\def\footnote{\protect\rmarkdownfootnote}

%%% Change title format to be more compact
\usepackage{titling}

% Create subtitle command for use in maketitle
\newcommand{\subtitle}[1]{
  \posttitle{
    \begin{center}\large#1\end{center}
    }
}

\setlength{\droptitle}{-2em}
  \title{Week 1 - Homework}
  \pretitle{\vspace{\droptitle}\centering\huge}
  \posttitle{\par}
  \author{STAT 420, Summer 2018, Unger}
  \preauthor{\centering\large\emph}
  \postauthor{\par}
  \date{}
  \predate{}\postdate{}


\begin{document}
\maketitle

\begin{center}\rule{0.5\linewidth}{\linethickness}\end{center}

Student Name: Hussein Ahmed Elmessilhy NetID: hae2

\subsection{Exercise 1 (Subsetting and
Statistics)}\label{exercise-1-subsetting-and-statistics}

For this exercise, we will use the \texttt{msleep} dataset from the
\texttt{ggplot2} package.

\textbf{(a)} Install and load the \texttt{ggplot2} package. \textbf{Do
not} include the installation command in your \texttt{.Rmd} file. (If
you do it will install the package every time you knit your file.)
\textbf{Do} include the command to load the package into your
environment.

\begin{Shaded}
\begin{Highlighting}[]
\KeywordTok{library}\NormalTok{(ggplot2)}
\end{Highlighting}
\end{Shaded}

\textbf{(b)} Note that this dataset is technically a \texttt{tibble},
not a data frame. How many observations are in this dataset? How many
variables? What are the observations in this dataset?

\textbf{Answer}: As viewed in the tibble of the dataset below, there are
83 observations and 11 variables. The observations represent animals.

\begin{Shaded}
\begin{Highlighting}[]
\CommentTok{#msleep <- ggplot2::msleep}
\NormalTok{msleep}
\end{Highlighting}
\end{Shaded}

\begin{verbatim}
## # A tibble: 83 x 11
##    name   genus vore  order conservation sleep_total sleep_rem sleep_cycle
##    <chr>  <chr> <chr> <chr> <chr>              <dbl>     <dbl>       <dbl>
##  1 Cheet~ Acin~ carni Carn~ lc                 12.1     NA          NA    
##  2 Owl m~ Aotus omni  Prim~ <NA>               17.0      1.80       NA    
##  3 Mount~ Aplo~ herbi Rode~ nt                 14.4      2.40       NA    
##  4 Great~ Blar~ omni  Sori~ lc                 14.9      2.30        0.133
##  5 Cow    Bos   herbi Arti~ domesticated        4.00     0.700       0.667
##  6 Three~ Brad~ herbi Pilo~ <NA>               14.4      2.20        0.767
##  7 North~ Call~ carni Carn~ vu                  8.70     1.40        0.383
##  8 Vespe~ Calo~ <NA>  Rode~ <NA>                7.00    NA          NA    
##  9 Dog    Canis carni Carn~ domesticated       10.1      2.90        0.333
## 10 Roe d~ Capr~ herbi Arti~ lc                  3.00    NA          NA    
## # ... with 73 more rows, and 3 more variables: awake <dbl>, brainwt <dbl>,
## #   bodywt <dbl>
\end{verbatim}

\textbf{(c)} What is the mean hours of REM sleep of individuals in this
dataset?

\textbf{Answer}:

\begin{Shaded}
\begin{Highlighting}[]
\KeywordTok{mean}\NormalTok{(msleep}\OperatorTok{$}\NormalTok{sleep_rem, }\DataTypeTok{na.rm =} \OtherTok{TRUE}\NormalTok{)}
\end{Highlighting}
\end{Shaded}

\begin{verbatim}
## [1] 1.87541
\end{verbatim}

\textbf{(d)} What is the standard deviation of brain weight of
individuals in this dataset?

\textbf{Answer}:

\begin{Shaded}
\begin{Highlighting}[]
\KeywordTok{sd}\NormalTok{(msleep}\OperatorTok{$}\NormalTok{brainwt, }\DataTypeTok{na.rm =} \OtherTok{TRUE}\NormalTok{)}
\end{Highlighting}
\end{Shaded}

\begin{verbatim}
## [1] 0.9764137
\end{verbatim}

\textbf{(e)} Which observation (provide the \texttt{name}) in this
dataset gets the most REM sleep?

\textbf{Answer}:

\begin{Shaded}
\begin{Highlighting}[]
\NormalTok{msleep[}\KeywordTok{which.max}\NormalTok{(msleep}\OperatorTok{$}\NormalTok{sleep_rem),]}\OperatorTok{$}\NormalTok{name}
\end{Highlighting}
\end{Shaded}

\begin{verbatim}
## [1] "Thick-tailed opposum"
\end{verbatim}

\textbf{(f)} What is the average bodyweight of carnivores in this
dataset?

\textbf{Answer}:

\begin{Shaded}
\begin{Highlighting}[]
\KeywordTok{mean}\NormalTok{(msleep[msleep}\OperatorTok{$}\NormalTok{vore }\OperatorTok{==}\StringTok{ "carni"}\NormalTok{,]}\OperatorTok{$}\NormalTok{sleep_rem, }\DataTypeTok{na.rm =} \OtherTok{TRUE}\NormalTok{)}
\end{Highlighting}
\end{Shaded}

\begin{verbatim}
## [1] 2.29
\end{verbatim}

\begin{center}\rule{0.5\linewidth}{\linethickness}\end{center}

\subsection{Exercise 2 (Plotting)}\label{exercise-2-plotting}

For this exercise, we will use the \texttt{birthwt} dataset from the
\texttt{MASS} package.

\textbf{(a)} Note that this dataset is a data frame and all of the
variables are numeric. How many observations are in this dataset? How
many variables? What are the observations in this dataset?

\textbf{Answer}: There are 189 observations and 10 variables in this
dataset, the observations represent infant birth weights along with the
risk factors associated with it

\begin{Shaded}
\begin{Highlighting}[]
\KeywordTok{library}\NormalTok{(MASS)}
\KeywordTok{str}\NormalTok{(birthwt)}
\end{Highlighting}
\end{Shaded}

\begin{verbatim}
## 'data.frame':    189 obs. of  10 variables:
##  $ low  : int  0 0 0 0 0 0 0 0 0 0 ...
##  $ age  : int  19 33 20 21 18 21 22 17 29 26 ...
##  $ lwt  : int  182 155 105 108 107 124 118 103 123 113 ...
##  $ race : int  2 3 1 1 1 3 1 3 1 1 ...
##  $ smoke: int  0 0 1 1 1 0 0 0 1 1 ...
##  $ ptl  : int  0 0 0 0 0 0 0 0 0 0 ...
##  $ ht   : int  0 0 0 0 0 0 0 0 0 0 ...
##  $ ui   : int  1 0 0 1 1 0 0 0 0 0 ...
##  $ ftv  : int  0 3 1 2 0 0 1 1 1 0 ...
##  $ bwt  : int  2523 2551 2557 2594 2600 2622 2637 2637 2663 2665 ...
\end{verbatim}

\textbf{(b)} Create a scatter plot of birth weight (y-axis) vs mother's
weight before pregnancy (x-axis). Use a non-default color for the
points. (Also, be sure to give the plot a title and label the axes
appropriately.) Based on the scatter plot, does there seem to be a
relationship between the two variables? Briefly explain.

\textbf{Answer}: There's no clear relationship between the two
variables, and that's because some mothers have above average weight and
have below average baby and vice versa. The graph looks more like noise.

\begin{Shaded}
\begin{Highlighting}[]
\KeywordTok{plot}\NormalTok{(bwt}\OperatorTok{~}\NormalTok{lwt, }\DataTypeTok{data =}\NormalTok{ birthwt,}
     \DataTypeTok{xlab =} \StringTok{"Mother last weight"}\NormalTok{,}
     \DataTypeTok{ylab =} \StringTok{"Baby Birth Weight"}\NormalTok{,}
     \DataTypeTok{main =} \StringTok{"Relationship between baby and mother weight"}\NormalTok{,}
     \DataTypeTok{col =} \StringTok{"orange"}\NormalTok{)}
\end{Highlighting}
\end{Shaded}

\includegraphics{w01-hw-hae2_files/figure-latex/unnamed-chunk-7-1.pdf}

\textbf{(c)} Create a scatter plot of birth weight (y-axis) vs mother's
age (x-axis). Use a non-default color for the points. (Also, be sure to
give the plot a title and label the axes appropriately.) Based on the
scatter plot, does there seem to be a relationship between the two
variables? Briefly explain.

\textbf{Answer}: Although there's no clear relationship beween the
mother age and the baby weight, but a lot of the anomalies (very low
weight or very high weight) are associated with older moms (above 25)

\begin{Shaded}
\begin{Highlighting}[]
\KeywordTok{plot}\NormalTok{(bwt}\OperatorTok{~}\NormalTok{age, }\DataTypeTok{data =}\NormalTok{ birthwt,}
     \DataTypeTok{xlab =} \StringTok{"Mother age"}\NormalTok{,}
     \DataTypeTok{ylab =} \StringTok{"Baby Birth Weight"}\NormalTok{,}
     \DataTypeTok{main =} \StringTok{"Relationship between baby weight and mother age"}\NormalTok{,}
     \DataTypeTok{col =} \StringTok{"orange"}\NormalTok{)}
\end{Highlighting}
\end{Shaded}

\includegraphics{w01-hw-hae2_files/figure-latex/unnamed-chunk-8-1.pdf}

\textbf{(d)} Create side-by-side boxplots for birth weight grouped by
smoking status. Use non-default colors for the plot. (Also, be sure to
give the plot a title and label the axes appropriately.) Based on the
boxplot, does there seem to be a difference in birth weight for mothers
who smoked? Briefly explain.

\textbf{Answer}: There's some difference between the mothers who smoke
and who haven't, mothers who smoke seem to have babies with a bit lower
weight than those who haven't as the median, 2nd, and 3rd percentiles
are lower. Difference is not that big though.

\begin{Shaded}
\begin{Highlighting}[]
\KeywordTok{boxplot}\NormalTok{(bwt}\OperatorTok{~}\NormalTok{smoke, }\DataTypeTok{data =}\NormalTok{ birthwt,}
     \DataTypeTok{xlab =} \StringTok{"Mother smoke status"}\NormalTok{,}
     \DataTypeTok{ylab =} \StringTok{"Baby Birth Weight"}\NormalTok{,}
     \DataTypeTok{main =} \StringTok{"Relationship between baby weight and mother smoke status"}\NormalTok{,}
     \DataTypeTok{col =} \StringTok{"orange"}\NormalTok{,}
     \DataTypeTok{names =} \KeywordTok{c}\NormalTok{(}\StringTok{"Non-Smoker"}\NormalTok{,}\StringTok{"Smoker"}\NormalTok{))}
\end{Highlighting}
\end{Shaded}

\includegraphics{w01-hw-hae2_files/figure-latex/unnamed-chunk-9-1.pdf}

\begin{center}\rule{0.5\linewidth}{\linethickness}\end{center}

\subsection{Exercise 3 (Importing Data, More
Plotting)}\label{exercise-3-importing-data-more-plotting}

For this exercise we will use the data stored in
\href{nutrition-2018.csv}{\texttt{nutrition-2018.csv}}. It contains the
nutritional values per serving size for a large variety of foods as
calculated by the USDA in 2018. It is a cleaned version totaling 5956
observations and is current as of April 2018.

\begin{Shaded}
\begin{Highlighting}[]
\KeywordTok{setwd}\NormalTok{(}\StringTok{".}\CharTok{\textbackslash{}\textbackslash{}}\StringTok{"}\NormalTok{)}
\NormalTok{food <-}\StringTok{ }\KeywordTok{read.csv}\NormalTok{(}\StringTok{".}\CharTok{\textbackslash{}\textbackslash{}}\StringTok{nutrition-2018.csv"}\NormalTok{)}
\end{Highlighting}
\end{Shaded}

The variables in the dataset are:

\begin{itemize}
\tightlist
\item
  \texttt{ID}
\item
  \texttt{Desc} - short description of food
\item
  \texttt{Water} - in grams
\item
  \texttt{Calories} - in kcal
\item
  \texttt{Protein} - in grams
\item
  \texttt{Fat} - in grams
\item
  \texttt{Carbs} - carbohydrates, in grams
\item
  \texttt{Fiber} - in grams
\item
  \texttt{Sugar} - in grams
\item
  \texttt{Calcium} - in milligrams
\item
  \texttt{Potassium} - in milligrams
\item
  \texttt{Sodium} - in milligrams
\item
  \texttt{VitaminC} - vitamin C, in milligrams
\item
  \texttt{Chol} - cholesterol, in milligrams
\item
  \texttt{Portion} - description of standard serving size used in
  analysis
\end{itemize}

\textbf{(a)} Create a histogram of \texttt{Calories}. Do not modify
\texttt{R}'s default bin selection. Make the plot presentable. Describe
the shape of the histogram. Do you notice anything unusual?

\textbf{Answer}: Most of the observed foods have less calories, i.e.,
foods with less calories are more common than foods with higher calories

\begin{Shaded}
\begin{Highlighting}[]
\KeywordTok{hist}\NormalTok{(food}\OperatorTok{$}\NormalTok{Calories,}
     \DataTypeTok{xlab =} \StringTok{"Amount of calories"}\NormalTok{,}
     \DataTypeTok{ylab =} \StringTok{"Frequency"}\NormalTok{,}
     \DataTypeTok{main =} \StringTok{"Histogram of calories in different food"}\NormalTok{,}
     \DataTypeTok{col =} \StringTok{"orange"}\NormalTok{)}
\end{Highlighting}
\end{Shaded}

\includegraphics{w01-hw-hae2_files/figure-latex/unnamed-chunk-11-1.pdf}

\textbf{(b)} Create a scatter plot of calories (y-axis) vs protein
(x-axis). Make the plot presentable. Do you notice any trends? Do you
think that knowing only the protein content of a food, you could make a
good prediction of the calories in the food?

\textbf{Answer}: There's a clear trend in the data, the higher the
protein, the less variant the food calories, this means that it's easy
to predict the calories a food has given it has a high protein, however,
a food with low or no protein could be with high o rlow calories.
Therefore, we can't always predict the calories given the protein.

\begin{Shaded}
\begin{Highlighting}[]
\KeywordTok{plot}\NormalTok{(Calories}\OperatorTok{~}\NormalTok{Protein, }\DataTypeTok{data =}\NormalTok{ food,}
     \DataTypeTok{xlab =} \StringTok{"Protein"}\NormalTok{,}
     \DataTypeTok{ylab =} \StringTok{"Calories"}\NormalTok{,}
     \DataTypeTok{main =} \StringTok{"Relationship between Protein and Calories"}\NormalTok{,}
     \DataTypeTok{col =} \StringTok{"orange"}\NormalTok{)}
\end{Highlighting}
\end{Shaded}

\includegraphics{w01-hw-hae2_files/figure-latex/unnamed-chunk-12-1.pdf}

\textbf{(c)} Create a scatter plot of \texttt{Calories} (y-axis) vs
\texttt{4\ *\ Protein\ +\ 4\ *\ Carbs\ +\ 9\ *\ Fat} (x-axis). Make the
plot presentable. You will either need to add a new variable to the data
frame, or use the \texttt{I()} function in your formula in the call to
\texttt{plot()}. If you are at all familiar with nutrition, you may
realize that this formula calculates the calorie count based on the
protein, carbohydrate, and fat values. You'd expect then that the result
here is a straight line. Is it? If not, can you think of any reasons why
it is not?

\textbf{Answer}: The result is not strictly straight line, however, it's
close. I can think of a reason why it's not a straight line, maybe the
formula is not 100\% accurate, for example, maybe fiber, which is a
different kind of carbs, has different number of calories per gram.

\begin{Shaded}
\begin{Highlighting}[]
\KeywordTok{plot}\NormalTok{(Calories }\OperatorTok{~}\StringTok{ }\KeywordTok{I}\NormalTok{(}\DecValTok{4} \OperatorTok{*}\StringTok{ }\NormalTok{Protein }\OperatorTok{+}\StringTok{ }\DecValTok{4} \OperatorTok{*}\StringTok{ }\NormalTok{Carbs }\OperatorTok{+}\StringTok{ }\DecValTok{9} \OperatorTok{*}\StringTok{ }\NormalTok{Fat), }\DataTypeTok{data =}\NormalTok{ food,}
     \DataTypeTok{xlab =} \StringTok{"Mixed food formula"}\NormalTok{,}
     \DataTypeTok{ylab =} \StringTok{"Calories"}\NormalTok{,}
     \DataTypeTok{main =} \StringTok{"Relationship between Protein and Calories"}\NormalTok{,}
     \DataTypeTok{col =} \StringTok{"orange"}\NormalTok{)}
\end{Highlighting}
\end{Shaded}

\includegraphics{w01-hw-hae2_files/figure-latex/unnamed-chunk-13-1.pdf}

\begin{center}\rule{0.5\linewidth}{\linethickness}\end{center}

\subsection{Exercise 4 (Writing and Using
Functions)}\label{exercise-4-writing-and-using-functions}

For each of the following parts, use the following vectors:

\begin{Shaded}
\begin{Highlighting}[]
\NormalTok{a =}\StringTok{ }\DecValTok{1}\OperatorTok{:}\DecValTok{10}
\NormalTok{b =}\StringTok{ }\DecValTok{10}\OperatorTok{:}\DecValTok{1}
\NormalTok{c =}\StringTok{ }\KeywordTok{rep}\NormalTok{(}\DecValTok{1}\NormalTok{, }\DataTypeTok{times =} \DecValTok{10}\NormalTok{)}
\NormalTok{d =}\StringTok{ }\DecValTok{2} \OperatorTok{^}\StringTok{ }\NormalTok{(}\DecValTok{1}\OperatorTok{:}\DecValTok{10}\NormalTok{)}
\end{Highlighting}
\end{Shaded}

\textbf{(a)} Write a function called \texttt{sum\_of\_squares}.

\begin{itemize}
\tightlist
\item
  Arguments:

  \begin{itemize}
  \tightlist
  \item
    A vector of numeric data \texttt{x}
  \end{itemize}
\item
  Output:

  \begin{itemize}
  \tightlist
  \item
    The sum of the squares of the elements of the vector
    \(\sum_{i = 1}^n x_i^2\)
  \end{itemize}
\end{itemize}

Provide your function, as well as the result of running the following
code \textbf{Answer}:

\begin{Shaded}
\begin{Highlighting}[]
\NormalTok{sum_of_squares =}\StringTok{ }\ControlFlowTok{function}\NormalTok{(x)\{}
  \KeywordTok{return}\NormalTok{(}\KeywordTok{sum}\NormalTok{(x }\OperatorTok{^}\StringTok{ }\DecValTok{2}\NormalTok{))}
\NormalTok{\}}

\KeywordTok{sum_of_squares}\NormalTok{(}\DataTypeTok{x =}\NormalTok{ a)}
\end{Highlighting}
\end{Shaded}

\begin{verbatim}
## [1] 385
\end{verbatim}

\begin{Shaded}
\begin{Highlighting}[]
\KeywordTok{sum_of_squares}\NormalTok{(}\DataTypeTok{x =} \KeywordTok{c}\NormalTok{(c, d))}
\end{Highlighting}
\end{Shaded}

\begin{verbatim}
## [1] 1398110
\end{verbatim}

\textbf{(b)} Using only your function \texttt{sum\_of\_squares()},
\texttt{mean()}, \texttt{sqrt()}, and basic math operations such as
\texttt{+} and \texttt{-}, calculate

\[
\sqrt{\frac{1}{n}\sum_{i = 1}^n (x_i - 0)^{2}}
\]

where the \(x\) vector is \texttt{d} and the \(y\) vector is \texttt{b}.

\textbf{Answer}:

\begin{Shaded}
\begin{Highlighting}[]
\KeywordTok{sqrt}\NormalTok{(}\KeywordTok{mean}\NormalTok{(d) }\OperatorTok{*}\StringTok{ }\KeywordTok{sum_of_squares}\NormalTok{(d) }\OperatorTok{*}\StringTok{ }\NormalTok{(}\DecValTok{1} \OperatorTok{/}\StringTok{ }\KeywordTok{sum_of_squares}\NormalTok{(}\KeywordTok{sqrt}\NormalTok{(d))) )}
\end{Highlighting}
\end{Shaded}

\begin{verbatim}
## [1] 373.9118
\end{verbatim}

\textbf{(c)} Using only your function \texttt{sum\_of\_squares()},
\texttt{mean()}, \texttt{sqrt()}, and basic math operations such as
\texttt{+} and \texttt{-}, calculate

\[
\sqrt{\frac{1}{n}\sum_{i = 1}^n (x_i - y_i)^{2}}
\]

where the \(x\) vector is \texttt{a} and the \(y\) vector is \texttt{b}.

\textbf{Answer}:

\begin{Shaded}
\begin{Highlighting}[]
\KeywordTok{sqrt}\NormalTok{(}\KeywordTok{mean}\NormalTok{(a) }\OperatorTok{*}\StringTok{ }\KeywordTok{sum_of_squares}\NormalTok{(a }\OperatorTok{-}\StringTok{ }\NormalTok{b) }\OperatorTok{*}\StringTok{ }\NormalTok{(}\DecValTok{1} \OperatorTok{/}\StringTok{ }\KeywordTok{sum_of_squares}\NormalTok{(}\KeywordTok{sqrt}\NormalTok{(a))) )}
\end{Highlighting}
\end{Shaded}

\begin{verbatim}
## [1] 5.744563
\end{verbatim}

\begin{center}\rule{0.5\linewidth}{\linethickness}\end{center}

\subsection{Exercise 5 (More Writing and Using
Functions)}\label{exercise-5-more-writing-and-using-functions}

For each of the following parts, use the following vectors:

\begin{Shaded}
\begin{Highlighting}[]
\KeywordTok{set.seed}\NormalTok{(}\DecValTok{42}\NormalTok{)}
\NormalTok{x =}\StringTok{ }\DecValTok{1}\OperatorTok{:}\DecValTok{100}
\NormalTok{y =}\StringTok{ }\KeywordTok{rnorm}\NormalTok{(}\DecValTok{1000}\NormalTok{)}
\NormalTok{z =}\StringTok{ }\KeywordTok{runif}\NormalTok{(}\DecValTok{150}\NormalTok{, }\DataTypeTok{min =} \DecValTok{0}\NormalTok{, }\DataTypeTok{max =} \DecValTok{1}\NormalTok{)}
\end{Highlighting}
\end{Shaded}

\textbf{(a)} Write a function called \texttt{list\_extreme\_values}.

\begin{itemize}
\tightlist
\item
  Arguments:

  \begin{itemize}
  \tightlist
  \item
    A vector of numeric data \texttt{x}
  \item
    A positive constant, \texttt{k}, with a default value of \texttt{2}
  \end{itemize}
\item
  Output:

  \begin{itemize}
  \tightlist
  \item
    A list with two elements:

    \begin{itemize}
    \tightlist
    \item
      \texttt{small}, a vector of elements of \texttt{x} that are \(k\)
      sample standard deviations less than the sample mean. That is, the
      observations that are smaller than \(\bar{x} - k \cdot s\).
    \item
      \texttt{large}, a vector of elements of \texttt{x} that are \(k\)
      sample standard deviations greater than the sample mean. That is,
      the observations that are larger than \(\bar{x} + k \cdot s\).
    \end{itemize}
  \end{itemize}
\end{itemize}

Provide your function, as well as the result of running the following
code.

\textbf{Answer}:

\begin{Shaded}
\begin{Highlighting}[]
\NormalTok{list_extreme_values =}\StringTok{ }\ControlFlowTok{function}\NormalTok{(x, }\DataTypeTok{k =} \DecValTok{2}\NormalTok{)\{}
\NormalTok{  results =}\StringTok{ }\KeywordTok{list}\NormalTok{(}
    \DataTypeTok{small =}\NormalTok{ x[x }\OperatorTok{<}\StringTok{ }\KeywordTok{mean}\NormalTok{(x) }\OperatorTok{-}\StringTok{ }\NormalTok{(k }\OperatorTok{*}\StringTok{ }\KeywordTok{sd}\NormalTok{(x))],}
    \DataTypeTok{large =}\NormalTok{ x[x }\OperatorTok{>}\StringTok{ }\KeywordTok{mean}\NormalTok{(x) }\OperatorTok{+}\StringTok{ }\NormalTok{(k }\OperatorTok{*}\StringTok{ }\KeywordTok{sd}\NormalTok{(x))]}
\NormalTok{  )}
  \KeywordTok{return}\NormalTok{(results)}
\NormalTok{\}}
\KeywordTok{list_extreme_values}\NormalTok{(}\DataTypeTok{x =}\NormalTok{ x, }\DataTypeTok{k =} \DecValTok{1}\NormalTok{)}
\end{Highlighting}
\end{Shaded}

\begin{verbatim}
## $small
##  [1]  1  2  3  4  5  6  7  8  9 10 11 12 13 14 15 16 17 18 19 20 21
## 
## $large
##  [1]  80  81  82  83  84  85  86  87  88  89  90  91  92  93  94  95  96
## [18]  97  98  99 100
\end{verbatim}

\begin{Shaded}
\begin{Highlighting}[]
\KeywordTok{list_extreme_values}\NormalTok{(}\DataTypeTok{x =}\NormalTok{ y, }\DataTypeTok{k =} \DecValTok{3}\NormalTok{)}
\end{Highlighting}
\end{Shaded}

\begin{verbatim}
## $small
## [1] -3.371739
## 
## $large
## [1] 3.229069 3.211199 3.495304
\end{verbatim}

\begin{Shaded}
\begin{Highlighting}[]
\KeywordTok{list_extreme_values}\NormalTok{(}\DataTypeTok{x =}\NormalTok{ y, }\DataTypeTok{k =} \DecValTok{2}\NormalTok{)}
\end{Highlighting}
\end{Shaded}

\begin{verbatim}
## $small
##  [1] -2.656455 -2.440467 -2.414208 -2.993090 -2.699930 -2.113200 -2.188835
##  [8] -2.071388 -2.138368 -2.461335 -2.170247 -3.017933 -2.192786 -2.253132
## [15] -2.277778 -2.292971 -2.206485 -2.553825 -2.082814 -2.958780 -2.136025
## [22] -2.183149 -3.371739
## 
## $large
##  [1] 2.018424 2.286645 2.701891 2.059539 2.036972 2.049961 2.459594
##  [8] 2.212055 2.422163 2.019891 2.965865 2.098031 2.241904 2.041313
## [15] 3.229069 2.223534 3.211199 2.623495 2.727196 2.178668 3.495304
\end{verbatim}

\begin{Shaded}
\begin{Highlighting}[]
\KeywordTok{list_extreme_values}\NormalTok{(}\DataTypeTok{x =}\NormalTok{ z, }\DataTypeTok{k =} \FloatTok{1.5}\NormalTok{)}
\end{Highlighting}
\end{Shaded}

\begin{verbatim}
## $small
## [1] 0.001703130 0.077464589 0.047054933 0.060877148 0.009629518 0.004321658
## [7] 0.028495955 0.005327612 0.041129370
## 
## $large
##  [1] 0.9899656 0.9521815 0.9741261 0.9474009 0.9586979 0.9756436 0.9954564
##  [8] 0.9517322 0.9342643 0.9310075
\end{verbatim}

\textbf{(b)} Using only your function \texttt{list\_extreme\_values()},
\texttt{mean()}, and basic list operations, calculate the mean of
observations that are greater than 1.5 standard deviation above the mean
in the vector \texttt{y}.

\textbf{Answer}:

\begin{Shaded}
\begin{Highlighting}[]
\KeywordTok{mean}\NormalTok{(}\KeywordTok{list_extreme_values}\NormalTok{(y, }\FloatTok{1.5}\NormalTok{)}\OperatorTok{$}\NormalTok{large)}
\end{Highlighting}
\end{Shaded}

\begin{verbatim}
## [1] 1.970506
\end{verbatim}


\end{document}
